%\documentclass[10pt, conference, compsocconf]{IEEEtran}
\documentclass[a4paper,11pt]{article}

\usepackage{soul}
\usepackage{amsmath}
\usepackage{amssymb}
\usepackage{color}
\usepackage{graphicx}
\usepackage{setspace}
\usepackage{hyperref}


\pdfinfo{%
  /Title    (Lab 1: Sieve of Eratosthenes)
  /Author   (Max Clive, Hawk Weisman, Alden Page, Ryan Cambier)
  /Subject  (CS 220 Programming Languages Lab)
  /Keywords (Programming languages)
}

%\pdfpagewidth 8.5in %
%\pdfpageheight 11.0in %

%\addtolength{\oddsidemargin}{-.50in} %.875in
%\addtolength{\evensidemargin}{-.50in} %.875in
%\addtolength{\textwidth}{1.in} %1.75in
%\addtolength{\topmargin}{-.875in} %.875in
%\addtolength{\textheight}{1.in} %1.75in

%opening

\begin{titlepage}

\title{Computer Science 220 \\[20pt]
Programming Languages Concepts\\ [20pt]
\textbf{Lab Assignment 1: Sieve of Eratosthenes}}
\author{Group NullPointerException: \\ Max Clive, Hawk Weisman, Alden Page, Ryan Cambier}
\date{September 13, 2013 }


\end{titlepage}


\usepackage{listings}
\usepackage{color}

\definecolor{dkgreen}{rgb}{0,0.6,0}
\definecolor{gray}{rgb}{0.5,0.5,0.5}
\definecolor{mauve}{rgb}{0.58,0,0.82}

\lstset{frame=tb,
  language=Java,
  aboveskip=3mm,
  belowskip=3mm,
  showstringspaces=false,
  columns=flexible,
  basicstyle={\small\ttfamily},
  numbers=none,
  numberstyle=\tiny\color{gray},
  keywordstyle=\color{blue},
  commentstyle=\color{dkgreen},
  stringstyle=\color{mauve},
  breaklines=true,
  breakatwhitespace=true
  tabsize=3
}

\begin{document}

\maketitle

\section{Comparing Java and C}

	Overall, the implementations of the algorithm didn't differ much between either language. The most notable difference is the length of the programs – C's minimal syntax allows for a more concise program, although it comes at the cost of less built-in error handling and more difficult debugging. Theoretically, the C version of the same program should run at least an order of magnitude faster than the Java version, but speed isn't terribly relevant in a project of this nature and scale.
	With knowledge of C's “gotchas” (e.g. accessing outside of the boundaries of arrays, dereferencing null pointers, various other pointer-related disasters,) it is 
	In contrast, while Java's design attempts to force the user to consider every possible contingency through error handling and use approved design patterns, this robustness and reliability comes at the cost of much more cluttered code. C allows the developer to just “get it done,” trusting he or she to insert as much complexity as is necessary, no matter what the consequences that that may entail.

\section{Java Implementation of Sieve}

\begin{lstlisting}
 
import java.util.Scanner; 
 
public class NullPointerException_Sieve { 
	 
	private final int MAX_NUM = 50000; 
	private boolean[] primes = new boolean[50001]; 
	private int upper, lower; 
	Scanner scan = new Scanner(System.in); 
	 
	/** 
	Constructor for Sieve 
	*/ 
	public NullPointerException_Sieve() { 
		primes[0] = false; 
		primes[1] = false; 
		upper = MAX_NUM; 
		lower = 1; 
		for (int i = 1; i < primes.length; i++){ 
			primes[i] = true; 
		} 
	} 
	 
	/** 
	Implements the Sieve of Eratosthenes algorithm 
	*/ 
	public void processSieve(){ 
		primes[1] = false;						// 1 is not a prime number 
		for(int p = 2; p < 50000;) {					 
			for(int count = 2; count * p < 50000; count++) {			 
				primes[count * p] = false; 
			} 
			while(p < 50000 && primes[++p] == false);	// fast forward to the next prime number 
		} 
	} 
	 
	/** 
	Shows the set of sexy pairs 
	*/ 
	public void showPrimes() { 
		System.out.printf("Here are all of the sexy prime pairs in the range %d to %d \n", lower, upper); 
		int count = 0; 
		for (int i = lower; i <= upper; i++) { 
			if (primes[i] == true && i-6 > 0 && primes[i-6] == true) { 
				System.out.printf("%d and %d \n", i, (i-6)); 
				count++; 
			} 
		} 
		System.out.printf("There were %d sexy prime pairs displayed between %d and %d.\n", count, lower, upper); 
	} 
	 
	/** 
	gets lower and upper boundaries 
	*/ 
	public void getBoundaries() { 
		System.out.println("Please enter a lower boundary and an upper boundary and I will print all of the sexy prime pairs between those boundaries."); 
		do { 
			while (!(1 < upper && upper < 50000)) { 
				System.out.print("Please enter the upper boundary (must be between 1 and 50000):"); 
				try { 
					upper = Integer.parseInt(scan.nextLine()); 
				} 
				catch (Exception e) { 
					System.out.println("NaN"); 
				} 
			} 
			while (!(1 < lower && lower < 50000)) { 
				System.out.print("Please enter the lower boundary (must be between 1 and " + upper + "):"); 
				try { 
					lower = Integer.parseInt(scan.nextLine()); 
				} 
				catch (Exception e) { 
					System.out.println("NaN"); 
				} 
			}  
		} while (!(lower <= upper)); 
	} 
}
\end{lstlisting}


\section{Java Test Program}

\begin{lstlisting}

public class NullPointerException_SieveTest { 
	public static void main(String argv[]) { 
	NullPointerException_Sieve steve = new NullPointerException_Sieve(); 
	steve.processSieve(); 
	steve.showPrimes(); 
	steve.getBoundaries(); 
	steve.showPrimes(); 
	} 
}

\end{lstlisting}


\section{C Implementation of Sieve}

\begin{lstlisting}

#include <stdio.h> 
typedef struct Bounds Bounds; 
 
//should move function declarations to a .h file 
void processSieve(); 
int boundSizeCheck(int endpoint); 
 
struct Bounds { 
    int lowBound; 
    int highBound; 
}; 
 
Bounds b = {0,0}; 
 
void getBoundaries() { 
    printf("Enter the lower bound: "); 
    scanf("%d", &b.lowBound); 
    printf("Enter he upper bound: "); 
    scanf("%d", &b.highBound); 
     
    if (boundSizeCheck(b.lowBound)) || (boundSizeCheck(b.highBound)){ 
        printf("Please enter the boundaries again.") 
        getBoundaries(); 
    } 
} 
 
 
//Make sure input is of reasonable size 
int boundSizeCheck(int bound) { 
    if (bound > 50000 || bound < 1){ 
        printf("Inputs must be below 50000 and above 0.\n"); 
        return 1; 
    } 
    else return 0; 
} 
 
void processSieve() { 
    //p is the current prime number being evaluated 
    int p = 2; 
    int i, j; 
    //Populate our array with numbers 
    for (i = 0; i < ARRAYSIZE; i++){ 
        primes[i] = i;  
    } 
 
    //iterate through every element 
    for (i = p; p < ARRAYSIZE; i++){ 
        //if the element is prime, cross off its multiples 
        if (primes[i]){ 
            for (j = i + p; j < ARRAYSIZE; j = j + p){ 
                primes[j] = 0; 
            } 
        } 
        p++; 
    } 
}

\end{lstlisting}

% begin typing

\end{document}
